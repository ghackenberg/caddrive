\documentclass{IEEEtran}
\usepackage{graphicx}
\graphicspath{ {./graphics/} }
% abfangen: überall mal bulletpoints, referenzen, bilder einfügen (kann auch deutsch sein)
% abstract, titlel amschluss, sonst von oben nach unten schreiben
% lieber hinten dünner lassen, vorne ist wichtig für roten faden

% intro: problem beschreiben, relatedwork: belegen ,dass problem noch nicht glöst ist (kann pingpong sein)
% initiale idee -> related work -> problem umformulieren -> recherche erweitern -> problem klar -> problem noch nicht gelöst
% wichtig: struktur immer aktuell halten
% spellchecking: extension installieren für code

\title{TODO}
\author{TODO}

\begin{document}
    \maketitle

    \begin{abstract}
    hallo
    ~\cite{Hackenberg2014}~\cite{Legat2014}~\cite{Teufl2015}
\end{abstract}

\section{Introduction}
    \label{sec:introduction}


    \subsection*{Context}
    % 3-4 absätze längster teil von oben
    % umgebungsbedingungen in denen das problem besteht 
    % referenzen: trends, kunden immer individuellere produkte, immer kürzeren zyklen, agile auf kunden reagieren, kosten, effizienz, digitalisierung
    % für jeden absatz ein zentrales thema zb agilität, digitalisierung, lotsize one: losgröße1, individualisierung von kunden, 
    % ausgangspunkt: losgröße1, agilität für umsetzung, digitalisierung als enabler für agilität
    % engineer-to-order bereich vor losgröße oder statt losgröße andere begriffe: assemble-to-order, fabricate-to-order, configure-to-order
    
    %engineer to order und losgröße1 individualisierung
    Over time, customers demand more and more specific products at small batch sizes.
    Engineer to order is a part of the production chain in which a product is produced and customized according to exact specifications.
    Such customization is becoming more and more important to increase customer satisfaction and thus also to increase the value of the company.

    % agilität
    An important part of this individual production consists of regular meetings. Through constant exchange, customer requirements can be discussed and specified.
    Through agile product management is it possible
    to adjust the product in the developing process if necessary. 

    % digitalisierung als enabler für die agilität - 
    Digitization enables such agile product management, as this would be laborious and sometimes impossible without databases and digital spreadsheets.  

    \subsection*{Problem}
    % anforderungs management schwierig, läuft häufig in excel, bezug zu cad daten bzw ergebnissen, engeneering artefakten geht verloren
    % aufgabenplanung, zeitplanung, kunde hat keinen zugriff auf anforderungen (excelliste) deswegen schwierig dass entwickler mit kundne diskutieren können
    % zeitenpriorisierung anforderungspriorisierung transperenter entwicklungsfortschritt
    % nur ein bis zwei absatz

    Requirements management is complicated by the fact that CAD data and documentation are kept separate. By keeping the documentation in Excel lists or documents, some engineering artifacts are lost. 

    Since the customer does not have access to the requirements or the Excel lists, agreements, task planning and scheduling are difficult. 
    
    This lack of transparency in product development makes it difficult to implement agile product management.
    
    If the customer has no access to the development progress, there is no possibility of prioritizing times and requirements.
    
   

    \subsection*{Solution}
    % platform nach vorbild von github für cad kurzer abriss wie problem gelöst wird 
    % platform grob beschreiben mit funktionen
    % 5-6 sätze 1 absatz

    To address these issues, we developed a product development platform modeled after GitHub. CAD models can be uploaded to our platform. The platform offers a user management and the possibility to add members with different rights to a CAD model. Authorized members can upload different versions of the CAD model, just like on GitHub. Milestones can be created for each CAD model. Issues can be added for each milestone, on the basis of which changes can be made to the model. For each issue there is a discussion thread that can be opened and closed. Once all issues have been resolved, the respective milestone is considered completed. On the platform, the CAD model can be viewed in a 3D view and is connected to the integrated product management tools. 

   

    \subsection*{Outline}
    % struktur vom dokument kurz erklärt
    This article is structured as follows:
    In Section~\ref{sec:differentiation} we discuss related work on the problem defined previously.
    In Section~\ref{sec:contribution} we present our original solution to the problem at hand.
    Then, in Section~\ref{sec:evaluation} we evaluate our solution with respect to different criteria.
    Finally, in Section~\ref{sec:conclusion} we summarize our learnings and provide an outlook on future work.
    
    \section{Related work}
    \label{sec:differentiation}
    % onshape, recherche bei google schoolar (problem als suchbegriff: requirement management, engineer to order)
    % wichtig um deutlich zu machen dass unser ansatz neu und wichtig ist
    % zentral: richtige suchbegriffe rumprobieren (manchmal werden dafür andere begriffe verwendet)
    % suchbegriffe notieren, systematik erklären können, wie man zu den quellen gekommen ist 
    % von 10000 ergebnisse erste 10 setien durchgescannt, anhand der kurzbeschreibung: relevant, vlt relevant, nicht relevant
    % relevante reingeschaut und dann erst entschieden 
    % wir form verwenden aktiv form überwiegend verwenden
    % papers: titel in google schoolar eingeben
    \section{Tool goals and requirements}
    \label{sec:requirements} 

    Our main goal is to improve the information flow between customers, project managers, requirements engineers, and product designers, and to reduce inaccuracies and inconsistencies in product development.
    In particular, we want to make it easier for the customers to participate in product development processes, to understand the current state of running projects, and to provide valuable feedback for validation.
    At the same time we want to make it easier for project managers, requirements engineers, and product designers to share and validate their individual view points and, hence, build a mutual understanding faster.

    In a first step, we translated these rather abstract goals into more concrete functional and non-functional requirements:
    \begin{description}
        \item[RQ1] Customers as well as project managers, requirements engineers, and product designers shall be able to authenticate and shall have different permissions. 
        \item[RQ2] Project managers shall be able to start new projects and assign customers, requirements engineers and product designers to these projects.
        \item[RQ3] Product designers shall be able to upload CAD models and maintain version histories independent of the CAD tool vendor used.
        \item[RQ4] Customers, project managers, requirements engineers, and product designers shall be able to create and discuss design tasks to be completed.
        \item[RQ5] It shall be possible to refer to parts and assemblies of CAD models during discussion of design tasks to improve the quality of information exchange.
        \item[RQ6] Customers shall be able to define priorities and set deadlines for design task execution, as well as track progress of design task execution.
        \item[RQ7] Project managers, requirements engineers, and product designers shall be able to report progress of design task execution.
        \item[RQ8] It shall be possible to use any device from smartphone over tablet to laptop and desktop to access the features anywhere and anytime.
        \item[RQ9] It shall be possible to extend the user interface with support for virtual reality and augmented reality hardware in the future.
    \end{description}
    In a second step, we derived a concrete tool design from the previous list of requirements.

\section{Tool design}
    \label{sec:contribution}

    As explained previously, our tool design comprises three models: An integrated \textbf{data model}, a \textbf{permission model}, and an \textbf{interface model}.
    The integrated data model describes our representation of stakeholders, roles, CAD model revisions, design tasks, and project schedules.
    The permission model explains which changes the individual stakeholders can perform on top of the data model depending on his/her role.
    Finally, the interface model demonstrates how the previous models can be put into action practically.

    \subsection*{Data model}

    \begin{figure*}[ht]
        \centering
        \includegraphics[width=\textwidth]{entities-v5.png}
        \caption{Integrated data model for improved information exchange between customers, project managers, requirements engineers, and product designers}
        \label{fig: datamodel}
    \end{figure*}

    This section describes the integrated data model as shown in Figure~\ref{fig: datamodel}.
    The data model contains the following entities: \textit{User}, \textit{Product}, \textit{Version}, \textit{Issue}, \textit{Comment}, \textit{Milestone}, and \textit{Member}.
    Note that all entities share a common identifier attribute, which is used for referencing and linking entities.
    Furthermore, all entities share a common deleted attribute, which is used to hide entities from the users after removal.

    % User
    The \textit{User} entity represents the individual stakeholders, who access the tool during product development.
    The entity defines an email and a password attribute, which are used for authentication purposes.
    Additionally, the entity defines a name attribute, which is used for a human-readable identification of stakeholders.
    Similarly, the entity links an \textit{Image} file, which provides a human-interpretable visual representation of stakeholders (i.e.\ a profile picture).
    Furthermore, the entity defines a user manager and a product manager flag, which are used for permission control as explained later.
    
    % Product
    The \textit{Product} entity represents the individual products or product development projects managed with the tool.
    The entity is always linked to a \textit{User} entity, representing the stakeholder who created the product in the first place.
    Then, the entity defines a name attribute, which is used to identify the product in a human-readable manner.
    Furthermore, the entity defines a description attribute, which is used to explain the purpose of the product only briefly.

    % Member
    The \textit{Member} entity represents the permission to access certain product data through the tool.
    Consequently, the entity is linked to a \textit{Product} entity, representing the product for which access is granted.
    Also, the entity is linked to a \textit{User} entity, representing the stakeholder who is granted product access.
    Finally, the entity defines a \textit{role} attribute, which controls the permission level as explained later, and which can have one of three values: \textit{manager}, \textit{engineer}, or \textit{customer}.

    % Version
    The \textit{Version} entity represents a revision of the product design created by a product designer.
    The entity is linked to a \textit{Product} entity to identify the product, for which the design revision was created.
    Similarly, the entity is linked to a \textit{User} entity to identify the product designer, who was responsible for creating the design revision.
    Furthermore, the entity is linked to previous \textit{Version} entities to document the history of design revisions including version branching and merging.
    Then, the entity defines a major, a minor, and a patch number, which together represent a version number according to semantic versioning\footnote{\url{https://semver.org/}} practices.
    Also, the entity defined a description, which provides a human-readable explanation of the design changes that have been applied.
    Finally, the entity links a \textit{Model} file, which represents the actual design data including assembly and part structures as well as geometry and material information.
    
    % Model
    We work with a generic representation of \textit{Model} files, which is independent of the CAD tool vendor and data format used.
    We assume a model contains \textit{Node} objects, which carry all the engineering information created by the product designer.
    Furthermore, we assume a node defines a name attribute, which can be used as for human-readable identification of nodes.
    Additionally, we assume a node provides a type attribute, which can be used to distinguish different types of nodes.
    At the moment, we distinguish two types of nodes, namely \textit{Group} nodes and \textit{Mesh} nodes.
    Groups represent the assembly structures and provide links to child nodes being assembled, which can be both groups and meshes.
    Meshes, on the other hand, represent the atomic parts of the product design such as screws or rivets, and carry geometry and material information.
    Technically, we work with the glTF\footnote{\url{https://www.khronos.org/gltf/}} format for representing models including their node structures as well as their geometry and material information.
    However, the STEP\footnote{\url{https://www.iso.org/standard/66654.html}} format or the COLLADA\footnote{\url{https://www.khronos.org/collada/}} format could be used equally well, since they follow the same principles.
    In fact, even a vendor-specific data format could be used as long as appropriate parsers are available.

    % Milestone
    The \textit{Milestone} entity essentially represents project schedules including deadlines for the realization of design tasks.
    The entity always links a \textit{Product} entity to identify the product or project the milestone belongs to.
    Furthermore, the entity links a \textit{User} entity to identify the product manager, who created the milestone.
    Finally, the entity defines a start and an end date, which represent the time frame for working on the milestone and achieving its goals.

    % Issue
    The \textit{Issue} entity represents design tasks, which have to be performed by product designers during product development projects.
    The entity is always linked to a \textit{Product} entity to identify the product or project the issue belongs to.
    Furthermore, the entity is linked to multiple \textit{User} entities to identify the one stakeholder, who reported the issue in the first place, and to identify the product designers, who are responsible for issue resolution.
    Moreover, the entity can be linked to a \textit{Milestone} entity to identify the time frame, within which the issue should be resolved.
    Then, the entity defines a time attribute, which records the point in time when the issue was reported originally.
    Also, the entity defines a label attribute, which provides a human-readable summary of the issue.
    Additionally, the entity defines a text attribute, which provides a more detailed explanation of the design task including markdown-based\footnote{\url{https://www.markdownguide.org/}} \textit{Reference} objects as explained later.
    Finally, the entity defines a state attribute, which enables us to distinguish between open and closed design tasks.

    % Comment
    The \textit{Comment} entity represents discussions between stakeholders on issues or design tasks respectively.
    The entity links an \textit{Issue} entity to identify the issue or design task the comment belongs to.
    Furthermore, the entity links a \textit{User} entity to identify the stakeholder, who posted the comment.
    Then, the entity defines a time attribute, which records the point in time when the comment was posted.
    Additionally, the entity defines a text attribute, which contains the actual content of the comment including markdown-based \textit{Reference} objects as explained later.
    Finally, the entity defines an action attribute, which can be used to close or reopen an issue by posting a comment.

    % Reference
    As explained previously, we work with a markdown-based representation of \textit{Reference} objects, which can be contained in the description of \textit{Issue} entities as well as \textit{Comment} entities.
    These references can be used to refer to \textit{Node} objects, that are contained in the CAD models of design revisions for a particular product or project.
    Consequently, the specification and discussion of design tasks can be enriched with links to assembly structures and parts, which have been designed and delivered previously.

    \subsection*{Permission model}
    \label{subsec:permissionModel}
    The permission model offers a number of possible restrictions on the platform so that not every user has all freedoms. This is especially important when cooperating with customers. The customer should only have the possibility to evaluate existing products. For this he can see the products he is registered for and use the given product management functions depending on its member role \textit{customer}. Creating new users and products should only be possible by the managers of the respective product. They also organize the rights of each user. The permission model is divided into two levels. The first level is the global level and the second level is the product level. The global level uses a permission model and the product level a role model which is implicitly linked to the permission model. 
    This allows the first level to manage products and users and the role model to distribute more specific permissions to the individual products. This system is deeply integrated in the backend.

    \subsubsection*{Global level}
    The \textit{User} entity has the two attributes \textit{user management permission} and \textit{product management permission}. Only a user who has the user management permission can create or edit users. On the other hand every user can edit his own personal data. With the product management permission it is possible to create new products. If a new product is created, the respective user is automatically also a product member and receives the member role \textit{manager}. The permissions of each user can be adjusted afterwards. So it is possible to give multiple users the permissions for user management and product management. For example, a user and product manager can exist for each department. 

    \subsubsection*{Product level}
    For permission management at product level, three member roles are provided. These roles are: \textit{manager, engineer and customer}. The manager is the one who created the product and has the permission to add more members. So he can add more managers, who in turn have all the rights over the management of the product. This is useful when the product development covers several departments. The second role is the engineer. He is involved in the product development process. He has no rights to change the product description or its members. However, he can freely create and edit versions, issues and comments. The last role is the customer who has the possibility to observe the product development process. He can create issues and comments and so participate in the product development process. In the current version of the software the rights of the customer are still very strict. The permission system is implemented in such a way that it can be changed with few adjustments. If the customer needs writing permissions, this can be easily changed.
    
    \subsection*{Interface model} 
    The user interface provides a convenient way to interact with the functional model to display and modify data. The user interface offers a consistent design, which runs through the entire system.

    \subsubsection*{Product view}
    The \textit{product view} page lists all available products, for which the user has the permission to see them, in a table [see Fig. \ref{fig: startpage} on page~\pageref{fig: startpage}]. For each product in the table a preview is shown. The other columns show the attributes \textit{owner, name, description, versions, issues and members}. The X on the right provides the possibility to delete the corresponding product. The owner is the person who created the product. Name and description are defined when the product is created and can be changed later. The columns on the right show how many versions exist for this product, how many issues have been created and how many members have access to the product. By clicking on New product you get also to the ProductSettings view where you can add a new product when you have the product management permission. This button is only visible when the corresponding user has product management permission. The CAD model and version details will be added when the user creates a new version of a product.

    \begin{figure}[h]
        \centering
        \includegraphics[width=\columnwidth]{startpage.JPG}
        \caption{Product view}
        \label{fig: startpage}
    \end{figure}

    \subsubsection*{ProductVersion view}
    Clicking on a product takes you to the ProductVersion view. You can also use the toolbar to jump to other pages such as issues, milestones members or settings. The left side of the view shows the created versions. On the left side there is a tree structure similar to GitHub. This tree structure results from the chronological arrangement of the product versions. The lines of the structure show how the product versions are in relation to the respective base versions. In the middle is the corresponding version number with the owner of the version inclusive email and a short description. Each version offers a preview. By clicking on the respective version, the 3D view on the right side also changes and shows the selected model. The 3D view allows to rotate, move and zoom the model. 
    With a click on the \textit{New version} button you get to the ProductVersionSettings view where you can create new product versions. [see Fig. \ref{fig: versionsettingsview} on page~\pageref{fig: versionsettingsview}]. 

    \subsubsection*{ProductVersionSettings view}
    Here you can enter information for a new version and select an GLB file. Depending on the selected base versions, the ProductVersion view shows the new version with the corresponding new tree structure after pressing the Save button. It is also possible to select multible base versions to merge them into a new version.

    \begin{figure}[h]
        \centering
        \includegraphics[width=\columnwidth]{versionsettingsview.JPG}
        \caption{ProductVersionSettings view}
        \label{fig: versionsettingsview}
    \end{figure}

    \subsubsection*{ProductIssue view}
    By clicking on the Issues link, you access the ProductIssue view [see Fig. \ref{fig: issueview} on page~\pageref{fig: issueview}]. Here the created issues are displayed in a table. The two buttons Open Issues and Closed Issues can be used to filter the list accordingly. The table shows the reporter who created the issue, the associated label, the assignees and how many comments and marked parts are in the conversation channel. 
    All views with 3D View offer the possibility to select a desired version for viewing. In the version view the version can be clicked directly. In the other views the version can be selected via a dropdown menu. This menu is located in the upper left corner of the 3D View. 
    If the user hovers with the mouse over a part of the 3D model, this part gets highlighted. 
    When hovering over an issue, the 3D view shows those parts that have been referenced in the comments of the respective issue by highlighting them in red.
    When hovering over an issue the parts are only highlighted if the version on which the parts were selected was picked in the dropdown menu. Thats because the 3d view shows for each version only the markers that were created on corresponding version.

    \begin{figure}[h]
        \centering
        \includegraphics[width=\columnwidth]{issueviewselectedpart.png}
        \caption{Issue view}
        \label{fig: issueview}
    \end{figure}

    \subsubsection*{ProductIssueSettings view}
    The ProductIssueSettings view allows to create new issues for the product [see Fig. \ref{fig: issuesettingsview} on page~\pageref{fig: issuesettingsview}]. The label, the text, the milestone and the assignees can be defined. An existing milestone can be selected with the dropdown menu. An issue must not be assigned to a milestone. This choice lies by the user. 
    As in the ProductIssue view, the model of the desired version can also be selected here via dropdown menu on the 3D view.
    The \textit{text} field in the ProductIssueSettings view represents the first comment in an issue. By clicking on the part, the part gets included in the comment as markdown text. If a comment includes a marked part when created, the marking of the respective part is saved to the associated product version.
    The Save button closes the settings, and you return to the ProductIssue view where the new issue is visible.

    \begin{figure}[h]
        \centering
        \includegraphics[width=\columnwidth]{issuesettingsview.JPG}
        \caption{Issuesettings view}
        \label{fig: issuesettingsview}
    \end{figure}

    \subsubsection*{ProductIssueComment view}
    Clicking on an issue in the Issue view opens the corresponding ProductIssueComment view.  Here you have the possibility to discuss the issue. 
    For this purpose, in the new comment box a text can be entered. This text field supports markdown.
    Like in the ProductIssueSettings view you can click on a part of the 3D model to mark a part as markdown[see Fig. \ref{fig: commentselectedpartview} on page~\pageref{fig: commentselectedpartview}]. In the course of a discussion, several parts can be marked in this way.
    A comment can also be used to close an issue by clicking the close button. This issue will then be found in Closed Issues in the ProductIssue view. With the comment function it is also possible to reopen the issue in the same way. The Close button displays the text Reopen when an issue is closed. In the upper right corner of the ProductIssueComment view there is a button to edit the selected issue. For example, the issue can be assigned to another Milestone or other attributes like the label, text or the list of assignees can be changed [see Fig. \ref{fig: issuesettingsview} on page~\pageref{fig: issuesettingsview}].

    \begin{figure}[h]
        \centering
        \includegraphics[width=\columnwidth]{commentselectedpartview.JPG}
        \caption{Selected part in ProductIssueComment view}
        \label{fig: commentselectedpartview}
    \end{figure}

    \subsubsection*{ProductMilestone view}
    The ProductMilestone view can be accessed via the Milestones link [see Fig. \ref{fig: milestoneview} on page~\pageref{fig: milestoneview}]. A table shows who created the milestone, its name, start date, end date and the progress. For each milestone two progress bars are displayed. The first one shows the date progress of a milestone by calculating \textit{100 * (nowDate - startDate) / (endDate - startDate)}. The second bar shows the issue progress of a milestone by calculating \textit{100 * number of closedIssues / number of allIssues}.
    A click on the New Milestone button leads to the ProductMilestoneSettings view. Here the attributes of a milestone can be adjusted and saved. The new or edited Milestone than show up in the ProductMilestone view.

    \begin{figure}[h]
        \centering
        \includegraphics[width=\columnwidth]{milestoneview.JPG}
        \caption{ProductMilestone view}
        \label{fig: milestoneview}
    \end{figure}

    \subsubsection*{ProductMilestoneIssue view}
    If a milestone is selected, a table with the attached issues is displayed [see Fig. \ref{fig: sprintview} on page~\pageref{fig: sprintview}]. This table is identical to the one in the ProductIssue view. Here you can also filter by open and closed issues. On the right side a burn down chart is displayed which shows the current progress of the milestone. The chart shows the start date, the end date, the number of issues and the progress until the current day. In the chart, the green line represents the target burndown and the blue graph the actual burndown. The target burndown distributes the open issues over the time span. The actual burndown drops by one for each issue that is closed.  Like in the ProductMilestone view, a click Edit Milestone button leads to the ProductMilestoneSettings view where the attributes of a milestone can be changed.

    \begin{figure}[h]
        \centering
        \includegraphics[width=\columnwidth]{sprintview.JPG}
        \caption{ProductMilestoneIssue view}
        \label{fig: sprintview}
    \end{figure}

    \subsubsection*{ProductMember view}
    To distribute the rights for a product, members are added to an existing product via the user interface. In the ProductMember view, a table shows all members who have access to the selected product [see Fig. \ref{fig: memberview} on page~\pageref{fig: memberview}]. The table shows the user picture and the name of the user. The role column defines which rights the respective member has. At the moment there are three roles: \textit{manager, engineer, customer} as explained in the permission model.. As with every overview table, objects can be deleted from the list by clicking on the X button. The button New Member leads to the ProductMemberSettings view where new members can be added.
    
    \begin{figure}[h]
        \centering
        \includegraphics[width=\columnwidth]{memberview.JPG}
        \caption{Member view}
        \label{fig: memberview}
    \end{figure}

    \section{Critical evaluation}
\label{sec:evaluation}
To solve the issues defined at the beginning of the article, we have developed the software ProductBoard. The software has not yet reached its final state but already meets a wide range of required issues. We have noticed a few points that are still insufficiently covered in the current software version or are still open questions.

\subsection*{Support for multiple file types}
At this stage, the tool only supports 3D CAD files in .GLB format. For more interdisciplinary use cases, the software needs to be modified to read and display exports in different formats. The software can also be extended to include image files, video files, or simulations. To do this, the reference mechanism in the data model must be modified so that the node can point to multiple models. For this we have already planned to let the data reference node inherit from other nodes that contain further file formats and will be implemented at a later time.

\subsection*{Bind management tools on products or versions}
During the test phase, we asked ourselves whether it is better for mechanical designs to tie the issues, comments and milestones to the respective product or to a product version. Both approaches have different advantages and disadvantages. Like the GitHub platform, our implementation binds the issues, comments and milestones to the respective product. This has the advantage that there is a better overall view of the product and the product can be managed version-independently. The disadvantage is that it can happen with the comments that components are referenced that no longer exist in the current product progress. The conversion, that the issues, comments and milestones are linked to the version would require the entire software to be rebuilt and is not planned after discussions.

\subsection*{Referencing components across multiple versions}
In the comments section, different components of the CAD model can be referenced in text and get highlighted. The problem is that it is possible that in the course of a discussion, components of different versions are referenced because the product changes in the course of the project. In the communication channel, however, only one version of the product is visible and thus only the components that have been referenced in this version can be seen. In the test phase it turned out that this is not a good solution. Here an overview must be built in the user interface, which can display all versions with the respective references so that the progress of the product is traceable.

\subsection*{3D view loads always the latest version}
If you switch between the different views in the user interface, for example from the issues to the milestones, the latest product version is always loaded and displayed in the 3D view on the right side. Therefore, in each view the version to be displayed must again be selected separately via the dropdown menu. Here the user interface must be modified in such a way that the last loaded version is displayed in each view in order to increase the user experience.

\subsection*{Rigid permission model cause problems}
The primitive role system that is currently implemented at the member level is not suitable for complex organizational structures. Here, the permission model and the data model must be modified so that the permissions can be fine-tuned. We see either the approach with several fixed member roles or with single permissions, which can be assigned to the respective member. In the user interface, the respective area must then also be designed so that the rights can be distributed more individually. In our opinion, the permissions on the user level are sufficient to be able to create users and products because the fine-tuning of the permissions then takes place via the member level.

\subsection*{The Tool is not optimized for mobile devices}
The platform is currently designed exclusively for PC screens and not intended as a mobile application, which is a major weakness. For this purpose, the user interface must be changed to a responsive design to look good on all devices. If the software becomes popular in the world of product development, a change to a responsive design will be necessary. For this purpose the user interface will have to be adapted to all possible end devices.


\subsection*{Asynchronous versus Synchronous collaboration}
Currently, asynchronous collaboration like on GitHub is possible using the tool. Setting up a synchronous collaboration within a company structure including the customer would bring many advantages, disadvantages and development effort with it. For synchronous collaboration, for example, care would have to be taken to ensure that user experience does not suffer if new content is constantly popping up at any point. Currently, the software is designed for asynchronous collaboration and whether a change will take place will be evaluated in the future.



\section{Conclusion}
\label{sec:conclusion}

\subsection*{Summary} 
The topic of product development and especially requirement engineering is very well covered by literature. Most articles deal with theoretical concepts on how product development can be improved. According to our literature research there are many excellent concepts but only few software tools that solve the problems practically. Thus, we have developed a software tool that covers aspects such as requirement engineering, project management and solution engineering to provide a coherent overview of product development. All relevant work results are clearly visible on one platform and can be managed. The software is lightweight, easy to learn. Exports from 3D CAD models can be uploaded and managed with our version management. With built-in tools such as Issues, Comments and Milestones, agile product design can be performed with the customer. We have tested and critically evaluated our software. Thereby a few questions and issues came up which we discussed for further improvements.

\subsection*{Outlook}
In the next steps, the issues from the critical evaluation will be addressed and improved. Furthermore, we have found out some more points to extend the platform practically. The following points will be discussed and implemented in the near future.

\subsection*{Support for multiple file types}
The software will be extended in the future so that different data exports can be uploaded. The view on the right side of the user interface can be used for all kinds of data. For example, we would like to make it possible to display video files, images and simulations like CFD of FEM directly in the tool. Videos and simulations of systems could be integrated directly into the tool and controlled there. Further media would be images or PDF files as well as two-dimensional construction drawings or electrical plans.

\subsection*{Test cases for requirements}
The requirement management is implemented in our tool via the issues and comments and enables the requirements to be presented in text form. An extension for the requirement management would be the implementation of automated test cases. By formalizing the requirements, test cases can be generated that automatically check for the fulfillment of the issues. Thus, the issues can be automatically closed or reopened depending on the test results.

\subsection*{Linking issues}
It can happen that two or more stakeholders create the same issues with a different wording. These issues then have similar or the same requirements, lie around on the platform and reduce the overview of the progress. One improvement is to make the issues linkable to each other. This way, duplicates can be found and then removed more easily.

\subsection*{Extract and display version deltas}
Another extension we would like to work on is calculating the deltas between the versions. So the user can see what has changed from one version to the other. The question is whether the delta is supplied by the CAD program or whether it has to be determined explicitly. If the delta has to be determined, an algorithm can be implemented that calculates the differences between the CAD data of the versions. Another approach would be that the user manually describes the differences when adding a version. However, this is a pragmatic solution and does not provide a good user experience. The user interface can then be adapted so that the version differences are clearly displayed.

\subsection*{Virtual Reality}
Virtual reality has a huge potential to display the CAD models even better. Through VR, the communication between the stackholders can be further improved because the work results can be better presented. Furthermore, the customer has the possibility to freely view complex components in the virtual environment and thus to better follow the product development and to better evaluate the progress.



    \bibliographystyle{plain}
    \bibliography{main}

    % papers: titel in google schoolar eingeben
\end{document}