\documentclass{IEEEtran}
% abfangen: überall mal bulletpoints, referenzen, bilder einfügen (kann auch deutsch sein)
% abstract, titlel amschluss, sonst von oben nach unten schreiben
% lieber hinten dünner lassen, vorne ist wichtig für roten faden

% intro: problem beschreiben, relatedwork: belegen ,dass problem noch nicht glöst ist (kann pingpong sein)
% initiale idee -> related work -> problem umformulieren -> recherche erweitern -> problem klar -> problem noch nicht gelöst
% wichtig: struktur immer aktuell halten
% spellchecking: extension installieren für code



\title{TODO}
\author{TODO}

\begin{document}
    \maketitle
 
    i want to eat a orange 

    \begin{abstract}
        TODO
        hallo
    \end{abstract}
    
    \section{Introduction}
    \label{sec:introduction}

    \subsection*{Context}
    TODO~\cite{Hackenberg2014}~\cite{Legat2014}~\cite{Teufl2015}
    % 3-4 absätze längster teil von oben
    % umgebungsbedingungen in denen das problem besteht 
    % referenzen: trends, kunden immer individuellere produkte, immer kürzeren zyklen, agile auf kunden reagieren, kosten, effizienz, digitalisierung
    % für jeden absatz ein zentrales thema zb agilität, digitalisierung, lotsize one: losgröße1, individualisierung von kunden, 
    % ausgangspunkt: losgröße1, agilität für umsetzung, digitalisierung als enabler für agilität
    % engineer-to-order bereich vor losgröße oder statt losgröße andere begriffe: assemble-to-order, fabricate-to-order, configure-to-order


    \subsection*{Problem}
    TODO 
    % anforderungs management schwierig, läuft häufig in excel, bezug zu cad daten bzw ergebnissen, engeneering artefakten geht verloren
    % aufgabenplanung, zeitplanung, kunde hat keinen zugriff auf anforderungen (excelliste) deswegen schwierig dass entwickler mit kundne diskutieren können
    % zeitenpriorisierung anforderungspriorisierung transperenter entwicklungsfortschritt
    % nur ein bis zwei absatz

    \subsection*{Solution}
    TODO 
    % platform nach vorbild von github für cad kurzer abriss wie problem gelöst wird 
    % platform grob beschreiben mit funktionen
    % 5-6 sätze 1 absatz

    \subsection*{Outline}
    % struktur vom dokument kurz erklärt
    This article is structured as follows:
    In Section~\ref{sec:differentiation} we discuss related work on the problem defined previously.
    In Section~\ref{sec:contribution} we present our original solution to the problem at hand.
    Then, in Section~\ref{sec:evaluation} we evaluate our solution with respect to different criteria.
    Finally, in Section~\ref{sec:conclusion} we summarize our learnings and provide an outlook on future work.
    
    \section{Related work}
    \label{sec:differentiation}
    TODO
    % onshape, recherche bei google schoolar (problem als suchbegriff: requirement management, engineer to order)
    % wichtig um deutlich zu machen dass unser ansatz neu und wichtig ist
    % zentral: richtige suchbegriffe rumprobieren (manchmal werden dafür andere begriffe verwendet)
    % suchbegriffe notieren, systematik erklären können, wie man zu den quellen gekommen ist 
    % von 10000 ergebnisse erste 10 setien durchgescannt, anhand der kurzbeschreibung: relevant, vlt relevant, nicht relevant
    % relevante reingeschaut und dann erst entschieden 
    % wir form verwenden aktiv form überwiegend verwenden

    \section{Our solution}
    \label{sec:contribution}
    TODO 
    % details architektur beschreiben, datenmodell, funktionsmodell, berechtigungsmodell, schnittstellenmodell(gui, )
    % verallgemeinerungen beschreiben, erkenntnisgewinn beschreiben der generiert worden ist, screenshots: modelle
    % subsections überlegen: struktur

    \section{Criticial evaluation}
    \label{sec:evaluation}
    TODO
    % stärken und schwächen, vor und nachteile, anhand von kriterien (noch überlegen), anwendbarkeit, vollständigkeit (wie vollständig wird definiertes problem gelöst)
    % wird das problem gelöst oder nicht systematisch runterbrechen und aufarbeiten: anwendbarketi (könnte das ein unternehmen heute schon so anwenden) 
    % welche probleme wurden nicht gelöst, warum wurdne sie nicht gelöst, wie könnt man sie grob lösen, ist lösung denkbar basierend auf aktuellen stand oder 
    % bedarf es einer kompletten überarbeitung

    \section{Conclusion}
    \label{sec:conclusion}
    % 

    \subsection*{Summary}
    TODO
    % ganzes paper zusammenfassen, grob, was wurde warum wie gemacht, ergebnisse, wichtigste erkenntnisse

    \subsection*{Outlook}
    TODO
    % next steps, was steht auf dem plan
    % vr, simulationsdaten, ecat, engineering-artefacts (arbeitsergebnisse einer engineering aktivität)

    \bibliographystyle{plain}
    \bibliography{main}

    % papers: titel in google schoolar eingeben
\end{document}