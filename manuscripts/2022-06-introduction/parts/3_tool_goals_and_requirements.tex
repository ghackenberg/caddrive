\section{Tool goals and requirements}
\label{sec:requirements}

Before designing a new tool for information exchange and collaboration between customers, project managers, requirements engineers, and product designer, we summarize our goals and derive concrete requirements.
We use the requirements to verify our tool design and the goals to validate the effectiveness of the underlying approach.

\subsection{Tool goals}

Our main goal is to improve the information flow between customers, project managers, requirements engineers, and product designers, and to reduce inaccuracies and inconsistencies in product development.
In particular, we want to make it easier for customers to participate in product development processes, to understand the current state of running projects, and to provide valuable feedback for validation.
At the same time we want to make it easier for project managers, requirements engineers, and product designers to share and validate their individual view points and, hence, build a mutual understanding faster.

\subsection{Tool requirements}

In a first step, we translated these rather abstract goals into more concrete functional and non-functional requirements:

\begin{enumerate}
    \item Customers as well as project managers, requirements engineers, and product designers shall be able to authenticate and shall have different permissions. 
    \item Project managers shall be able to start new projects and assign customers, requirements engineers and product designers to these projects.
    \item Product designers shall be able to upload CAD models and maintain version histories independent of the CAD tool vendor used.
    \item Customers, project managers, requirements engineers, and product designers shall be able to create and discuss design tasks to be completed.
    \item It shall be possible to refer to parts and assemblies of CAD models during discussion of design tasks to improve the quality of information exchange.
    \item Customers shall be able to define priorities and set deadlines for design task execution, as well as track progress of design task execution.
    \item Project managers, requirements engineers, and product designers shall be able to report progress of design task execution.
    \item It shall be possible to use any device from smartphone over tablet to laptop and desktop to access the features anywhere and anytime.
    \item It shall be possible to extend the user interface with support for virtual reality and augmented reality hardware in the future.
\end{enumerate}

In a second step, we derived a concrete tool design from the previous list of requirements.