\section{Conclusion}
\label{sec:conclusion}

\subsection*{Summary}

The topic of product development and especially requirement engineering is very well covered by literature. Most articles deal with theoretical concepts on how product development can be improved. According to our literature research there are many excellent concepts but only few software tools that solve the problems practically. Thus, we have developed a software tool that covers aspects such as version, issue und milestone management to provide an integrated platform for agile CAD. All relevant work results are clearly visible on one platform and can be managed. The software is lightweight and more specialized for agile CAD than the software \textit{onshape} due to its more specific scope of functions. Exports from 3D CAD models can be uploaded and managed with our version management. With built-in tools such as Issues, Comments and Milestones, agile product design can be performed with the customer. We have tested and critically evaluated our software. Thereby a few questions and issues came up which we discussed for further improvements.

\subsection*{Outlook}

In the next steps, the issues from the critical evaluation will be addressed and improved. Furthermore, we have found out some more points to extend the platform practically. The following points will be discussed and implemented in the near future.

\subsection*{Linking issues}

It can happen that two or more stakeholders create the same issues with a different wording. These issues then have similar or the same requirements, lie around on the platform and reduce transparency?. One improvement is to make the issues linkable to each other. This way, duplicates can be found and then marked.

\subsection*{Extract and display version deltas}

Another extension of the software would be that it can compute deltas between versions. So the user can see what has changed from one version to the other. The question is whether the delta is supplied by the CAD program or whether it has to be determined explicitly. If the delta has to be determined, an algorithm can be implemented that calculates the differences between the CAD data of the versions. Another approach would be that the user manually describes the differences when adding a version. However, this is a pragmatic solution and does not provide a good user experience. The user interface can then be adapted so that the version differences are clearly displayed.

\subsection*{Virtual Reality}

Virtual reality has a huge potential to display the CAD models even better. Through VR, the communication between the stackholders can be further improved because the work results can be better presented. Furthermore, the customer has the possibility to freely view complex components in the virtual environment and thus to better follow the product development and to better evaluate the progress.