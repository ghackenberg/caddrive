\begin{abstract}
    hallo
    ~\cite{Hackenberg2014}~\cite{Legat2014}~\cite{Teufl2015}
\end{abstract}

\section{Introduction}
    \label{sec:introduction}


    \subsection*{Context}
    % 3-4 absätze längster teil von oben
    % umgebungsbedingungen in denen das problem besteht 
    % referenzen: trends, kunden immer individuellere produkte, immer kürzeren zyklen, agile auf kunden reagieren, kosten, effizienz, digitalisierung
    % für jeden absatz ein zentrales thema zb agilität, digitalisierung, lotsize one: losgröße1, individualisierung von kunden, 
    % ausgangspunkt: losgröße1, agilität für umsetzung, digitalisierung als enabler für agilität
    % engineer-to-order bereich vor losgröße oder statt losgröße andere begriffe: assemble-to-order, fabricate-to-order, configure-to-order
    
    %engineer to order und losgröße1 individualisierung
    Over time, customers demand more and more specific products at small batch sizes.
    Engineer to order is a part of the production chain in which a product is produced and customized according to exact specifications.
    Such customization is becoming more and more important to increase customer satisfaction and thus also to increase the value of the company.

    % agilität
    An important part of this individual production consists of regular meetings. Through constant exchange, customer requirements can be discussed and specified.
    Through agile product management is it possible
    to adjust the product in the developing process if necessary. 

    % digitalisierung als enabler für die agilität - 
    Digitization enables such agile product management, as this would be laborious and sometimes impossible without databases and digital spreadsheets.  

    \subsection*{Problem}
    % anforderungs management schwierig, läuft häufig in excel, bezug zu cad daten bzw ergebnissen, engeneering artefakten geht verloren
    % aufgabenplanung, zeitplanung, kunde hat keinen zugriff auf anforderungen (excelliste) deswegen schwierig dass entwickler mit kundne diskutieren können
    % zeitenpriorisierung anforderungspriorisierung transperenter entwicklungsfortschritt
    % nur ein bis zwei absatz

    Requirements management is complicated by the fact that CAD data and documentation are kept separate. By keeping the documentation in Excel lists or documents, some engineering artifacts are lost. 

    Since the customer does not have access to the requirements or the Excel lists, agreements, task planning and scheduling are difficult. 
    
    This lack of transparency in product development makes it difficult to implement agile product management.
    
    If the customer has no access to the development progress, there is no possibility of prioritizing times and requirements.
    
   

    \subsection*{Solution}
    % platform nach vorbild von github für cad kurzer abriss wie problem gelöst wird 
    % platform grob beschreiben mit funktionen
    % 5-6 sätze 1 absatz

    To address these issues, we developed a product development platform modeled after GitHub. CAD models can be uploaded to our platform. The platform offers a user management and the possibility to add members with different rights to a CAD model. Authorized members can upload different versions of the CAD model, just like on GitHub. Milestones can be created for each CAD model. Issues can be added for each milestone, on the basis of which changes can be made to the model. For each issue there is a discussion thread that can be opened and closed. Once all issues have been resolved, the respective milestone is considered completed. On the platform, the CAD model can be viewed in a 3D view and is connected to the integrated product management tools. 

   

    \subsection*{Outline}
    % struktur vom dokument kurz erklärt
    This article is structured as follows:
    In Section~\ref{sec:differentiation} we discuss related work on the problem defined previously.
    In Section~\ref{sec:contribution} we present our original solution to the problem at hand.
    Then, in Section~\ref{sec:evaluation} we evaluate our solution with respect to different criteria.
    Finally, in Section~\ref{sec:conclusion} we summarize our learnings and provide an outlook on future work.
    
    \section{Related work}
    \label{sec:differentiation}
    % onshape, recherche bei google schoolar (problem als suchbegriff: requirement management, engineer to order)
    % wichtig um deutlich zu machen dass unser ansatz neu und wichtig ist
    % zentral: richtige suchbegriffe rumprobieren (manchmal werden dafür andere begriffe verwendet)
    % suchbegriffe notieren, systematik erklären können, wie man zu den quellen gekommen ist 
    % von 10000 ergebnisse erste 10 setien durchgescannt, anhand der kurzbeschreibung: relevant, vlt relevant, nicht relevant
    % relevante reingeschaut und dann erst entschieden 
    % wir form verwenden aktiv form überwiegend verwenden
    % papers: titel in google schoolar eingeben