\section{Critical evaluation}
\label{sec:evaluation}
To solve the issues defined at the beginning of the article, we have developed the software ProductBoard. The software has not yet reached its final state but already meets a wide range of required issues. We have noticed a few points that are still insufficiently covered in the current software version or are still open questions.

\subsection*{Support for multiple file types}
At this stage, the tool only supports 3D CAD files in .GLB format. For more interdisciplinary use cases, the software needs to be modified to read and display exports in different formats. The software can also be extended to include image files, video files, or simulations. To do this, the reference mechanism in the data model must be modified so that the node can point to multiple models. For this we have already planned to let the data reference node inherit from other nodes that contain further file formats and will be implemented at a later time.

\subsection*{Bind management tools on products or versions}
During the test phase, we asked ourselves whether it is better for mechanical designs to tie the issues, comments and milestones to the respective product or to a product version. Both approaches have different advantages and disadvantages. Like the GitHub platform, our implementation binds the issues, comments and milestones to the respective product. This has the advantage that there is a better overall view of the product and the product can be managed version-independently. The disadvantage is that it can happen with the comments that components are referenced that no longer exist in the current product progress. The conversion, that the issues, comments and milestones are linked to the version would require the entire software to be rebuilt and is not planned after discussions.

\subsection*{Referencing components across multiple versions}
In the comments section, different components of the CAD model can be referenced in text and get highlighted. The problem is that it is possible that in the course of a discussion, components of different versions are referenced because the product changes in the course of the project. In the communication channel, however, only one version of the product is visible and thus only the components that have been referenced in this version can be seen. In the test phase it turned out that this is not a good solution. Here an overview must be built in the user interface, which can display all versions with the respective references so that the progress of the product is traceable.

\subsection*{3D view loads always the latest version}
If you switch between the different views in the user interface, for example from the issues to the milestones, the latest product version is always loaded and displayed in the 3D view on the right side. Therefore, in each view the version to be displayed must again be selected separately via the dropdown menu. Here the user interface must be modified in such a way that the last loaded version is displayed in each view in order to increase the user experience.

\subsection*{Rigid permission model cause problems}
The primitive role system that is currently implemented at the member level is not suitable for complex organizational structures. Here, the permission model and the data model must be modified so that the permissions can be fine-tuned. We see either the approach with several fixed member roles or with single permissions, which can be assigned to the respective member. In the user interface, the respective area must then also be designed so that the rights can be distributed more individually. In our opinion, the permissions on the user level are sufficient to be able to create users and products because the fine-tuning of the permissions then takes place via the member level.

\subsection*{The Tool is not optimized for mobile devices}
The platform is currently designed exclusively for PC screens and not intended as a mobile application, which is a major weakness. For this purpose, the user interface must be changed to a responsive design to look good on all devices. If the software becomes popular in the world of product development, a change to a responsive design will be necessary. For this purpose the user interface will have to be adapted to all possible end devices.


\subsection*{Asynchronous versus Synchronous collaboration}
Currently, asynchronous collaboration like on GitHub is possible using the tool. Setting up a synchronous collaboration within a company structure including the customer would bring many advantages, disadvantages and development effort with it. For synchronous collaboration, for example, care would have to be taken to ensure that user experience does not suffer if new content is constantly popping up at any point. Currently, the software is designed for asynchronous collaboration and whether a change will take place will be evaluated in the future.



\section{Conclusion}
\label{sec:conclusion}

\subsection*{Summary} 
The topic of product development and especially requirement engineering is very well covered by literature. Most articles deal with theoretical concepts on how product development can be improved. According to our literature research there are many excellent concepts but only few software tools that solve the problems practically. Thus, we have developed a software tool that covers aspects such as requirement engineering, project management and solution engineering to provide a coherent overview of product development. All relevant work results are clearly visible on one platform and can be managed. The software is lightweight, easy to learn. Exports from 3D CAD models can be uploaded and managed with our version management. With built-in tools such as Issues, Comments and Milestones, agile product design can be performed with the customer. We have tested and critically evaluated our software. Thereby a few questions and issues came up which we discussed for further improvements.

\subsection*{Outlook}
In the next steps, the issues from the critical evaluation will be addressed and improved. Furthermore, we have found out some more points to extend the platform practically. The following points will be discussed and implemented in the near future.

\subsection*{Support for multiple file types}
The software will be extended in the future so that different data exports can be uploaded. The view on the right side of the user interface can be used for all kinds of data. For example, we would like to make it possible to display video files, images and simulations like CFD of FEM directly in the tool. Videos and simulations of systems could be integrated directly into the tool and controlled there. Further media would be images or PDF files as well as two-dimensional construction drawings or electrical plans.

\subsection*{Test cases for requirements}
The requirement management is implemented in our tool via the issues and comments and enables the requirements to be presented in text form. An extension for the requirement management would be the implementation of automated test cases. By formalizing the requirements, test cases can be generated that automatically check for the fulfillment of the issues. Thus, the issues can be automatically closed or reopened depending on the test results.

\subsection*{Linking issues}
It can happen that two or more stakeholders create the same issues with a different wording. These issues then have similar or the same requirements, lie around on the platform and reduce the overview of the progress. One improvement is to make the issues linkable to each other. This way, duplicates can be found and then removed more easily.

\subsection*{Extract and display version deltas}
Another extension we would like to work on is calculating the deltas between the versions. So the user can see what has changed from one version to the other. The question is whether the delta is supplied by the CAD program or whether it has to be determined explicitly. If the delta has to be determined, an algorithm can be implemented that calculates the differences between the CAD data of the versions. Another approach would be that the user manually describes the differences when adding a version. However, this is a pragmatic solution and does not provide a good user experience. The user interface can then be adapted so that the version differences are clearly displayed.

\subsection*{Virtual Reality}
Virtual reality has a huge potential to display the CAD models even better. Through VR, the communication between the stackholders can be further improved because the work results can be better presented. Furthermore, the customer has the possibility to freely view complex components in the virtual environment and thus to better follow the product development and to better evaluate the progress.

