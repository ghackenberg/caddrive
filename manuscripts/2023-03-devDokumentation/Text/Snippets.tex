\label{sec: Snippets}

\chapter{Snippets}
%-----------------------------------------------------------------------
% Aufzählung
%-----------------------------------------------------------------------
\section{Aufzählung}
\begin{itemize}
	\item Keine unbekannten Abkürzungen in Überschriften
	\item Keine Artikel (\textit{der, die, das}) am Anfang von Überschriften
	\item Kein Punkt am Ende von Überschriften
	\item Achten Sie darauf, dass keine Überschrift – ohne den zugehörigen Text – die letzte Zeile einer Seite bildet; es sollte auch keine Seite mit einer Einzelzeile beginnen, die zum letzten Absatz der vorhergehenden Seite gehört.
	\item Detaillierung entspricht den Schwerpunkten bzw. der Gewichtung der Arbeit
	\item Immer mindestens zwei Unterpunkte zu einem Gliederungspunkt, ansonsten nicht gliedern
	\item Verwenden Sie maximal 5 Gliederungsebenen	
\end{itemize}
Aufgrund der automatischen Gliederung von {\LaTeX} können die Überschriften hier nicht dargestellt, sondern nur beschrieben werden:
\begin{enumerate}
	\item	Gliederungsebene: \verb|\chapter{...}|-Befehl
	\item	Gliederungsebene: \verb|\section{...}|-Befehl
	\item	Gliederungsebene: \verb|\subsection{...}|-Befehl
	\item	Gliederungsebene: \verb|\subsubsection{...}|-Befehl
	\item	Gliederungsebene: \verb|\paragraph{...}|-Befehl
\end{enumerate}
%-----------------------------------------------------------------------
% Grafiken
%-----------------------------------------------------------------------
%% Neue Seite
\newpage
\section{Grafiken}
Damit die Grafiken richtig nummeriert werden und auch eine Referenz auf sie gesetzt werden kann, muss die Bild-Umgebung verwendet werden.\\
Mit folgendem Beispielcode kann eine Abbildung eingefügt werden:
\begin{figure}[h]
    \centering
    \includegraphics[width=0.50\textwidth]{FH.png}
    \caption{Beispiel für eine Abbildung}
    \label{fig: Beispielabbildung}
\end{figure}\\
Zwei Abbildungen nebeneinander, mit einer Beschriftung, können wie folgt eingefügt werden:
\begin{figure}[htp!]
    \begin{minipage}[t]{0.49\textwidth}
        \centering
        \includegraphics[width=0.49\textwidth]{FH.png}
    \end{minipage}
    \hfill
    \begin{minipage}[t]{0.49\textwidth}
        \centering
        \includegraphics[width=0.49\textwidth]{FH.png}
    \end{minipage}
    \caption{Beispiel für 2 Abbildungen nebeneinander, 1 Beschriftung}
    \label{fig: Beispielabbildung-2}
\end{figure}\\
Zwei Abbildungen nebeneinander, mit getrennter Beschriftung, können wie folgt eingefügt werden:
\begin{figure}[htp!]
    \begin{minipage}[t]{0.49\textwidth}
        \centering
        \includegraphics[width=0.49\textwidth]{FH.png}
        \caption{Beispiel für 2 Abbildungen nebeneinander, 2 Beschriftungen}
        \label{fig: Beispielabbildung-3}
    \end{minipage}
    \hfill
    \begin{minipage}[t]{0.49\textwidth}
        \centering
        \includegraphics[width=0.49\textwidth]{FH.png}
        \caption{Beispiel für 2 Abbildungen nebeneinander, 2 Beschriftungen}
        \label{fig: Beispielabbildung-4}
    \end{minipage}
\end{figure}\\

%-----------------------------------------------------------------------
% Tabellen
%-----------------------------------------------------------------------
%% Neue Seite
\newpage
\section{Tabellen}
Tabellen sollten im Text referenziert \bzw mit dem Text in Bezug gebracht werden [siehe Tabelle \ref{tab: Testtabelle} auf S.~\pageref{tab: Testtabelle}].\\
Für die Tabellen gibt es eine eigene Tabellen-Umgebung damit die Nummerierung an die richtige Position gesetzt wird und auch auf sie referenziert werden kann.\\
Die Tabelle selbst kann auf zwei unterschiedlichen Varianten eingefügt werden.
\subsection{Als Bild}
In der Tabellen-Umgebung kann ein Bild einer Tabelle eingefügt werden.
\begin{table}[h]
	\centering
	\caption{Testtabelle}
	\includegraphics[scale=0.6]{0000_Testtabelle.png}
	\label{tab: Testtabelle} 
\end{table}
\subsection{Als Tabelle}
Es können auch direkt im {\LaTeX} Tabellen erstellt werden.
\begin{table}[h]
	\centering
	\caption{Testtabelle 2}
	\begin{tabular}{c|c||c|}
		Spaltenüberschrift & Spaltenüberschrift & Spaltenüberschrift \\
		\midrule[2pt]
		1 & Test-1 & Test-2\\ 
		\hline 2 & 0 & 0  \\ 
		3 & 0 & 0  \\ 
		4 & 0 & 1  \\
		\hline 
	\end{tabular}
	\label{tab: Testtabelle-2} 
\end{table}


%-----------------------------------------------------------------------
% Formeln
%-----------------------------------------------------------------------
\newpage
\section{Formeln und Gleichungen}
Formeln und Gleichungen sind grundsätzlich in einen Satz integriert. Sie können direkt in den Fließtext, zum Beispiel in der Form $e=m\cdot c^2$, oder abgesetzt in eine eigene Zeile/einen eigenen Absatz geschrieben werden.
Im letzteren Fall kann eine Formel, zum Beispiel
%
\begin{align}
	e=m\cdot c^2,
	\label{eq: Formel1}
\end{align}
%
mit einer Nummer versehen werden, auf die im Text an späterer Stelle verwiesen wird (siehe Gleichung~\ref{eq: Formel1} auf S.~\pageref{eq: Formel1}).
\vspace{6pt}\\
Weitere Beispiele für die Einbettung von Formeln:
\\
Das Newtonsche Gesetz $F=ma$ ist eines der wichtigsten Naturgesetze der Physik.
\\
Die allgemeine Form einer Fourier-Reihe lautet
%
\begin{align}
	f(x)=a_0 +\sum_{n=1}^\infty \left( a_n \cos\frac{n\pi x}{L} + b_n \sin\frac{n\pi x}{L} \right) .
	\label{eq: Formel2}
\end{align}
%
Darin bezeichnet man $a_n$ und $b_n$ als Fourier-Koeffizienten. Man beachte dabei die Satzzeichen
in Gleichung~\ref{eq: Formel2} und in Gleichung~\ref{eq: Formel1}!

%-----------------------------------------------------------------------
% Zitieren
%-----------------------------------------------------------------------
\newpage
\section{Zitieren}
Beim Zitieren können Sie eine der folgenden Vorgehensweisen frei wählen:
\begin{itemize}
	\item rein manuelles Zitieren;
	\item Zitieren mithilfe eines Literaturverwaltungsprogramms.
\end{itemize}
Entscheidend ist, dass Sie die Quellenangaben und das Literaturverzeichnis Ihrer wissenschaftlichen Arbeit richtig gestalten, \dah die Quellenangaben gemäß Abschnitt \ref{sec: ZitierStil} auf S. \pageref{sec: ZitierStil} und das Literaturverzeichnis gemäß Abschnitt \ref{sec: Bibliography} auf S. \pageref{sec: Bibliography}.\\
Stimmen Sie bereits zu Beginn Ihrer Arbeit die Zitierweise und etwaige zu verwendende Zitiervorlagen mit Ihrem*Ihrer Betreuer*in ab. Die Verwendung von Zitiervorlagen erfordert unter Umständen manuelle Anpassungen vor Abgabe Ihrer wissenschaftlichen Arbeit.\\
Für das Literaturverzeichnis in {\LaTeX} wird eine *.bib-Datei benötigt. Es gibt mehrere Möglichkeiten, diese Datei zu erstellen (manuell, JabRef, \dots).

\subsection{Wichtige Hinweise zu Zitaten}
\begin{itemize}
	\item	Es werden wörtliche Zitate (grundsätzlich in Anführungszeichen) und sinngemäße Zitate (ohne Anführungszeichen) differenziert.
	\begin{addmargin}[10pt]{0pt}
		{\footnotesize Jedes Zitat muss überprüfbar und einwandfrei nachvollziehbar sein. Einwandfreies Zitieren ist Ausdruck wissenschaftlicher Sorgfalt. Übernommenes fremdes Gedankengut ist in jedem Fall – egal ob als wörtliches oder sinngemäßes Zitat – als solches kenntlich zu machen.\footnote{Karmasin/Ribing, 2019, S. 114.}}
	\end{addmargin}
	
	\item	Integrieren Sie kurze wörtliche Zitate in den Fließtext. Verwenden Sie bei Zitaten, die in Standardformatierung länger als drei Zeilen sind, folgende Vorlage. In diesem Fall verwenden Sie keine Anführungszeichen:
	\begin{addmargin}[10pt]{0pt}
		{\footnotesize Lorem ipsum dolor sit amet, consectetuer adipiscing elit. Maecenas porttitor congue massa. Fusce posuere, magna sed pulvinar ultricies, purus lectus malesuada libero, sit amet commodo magna eros quis urna. Nunc viverra imperdiet enim. Fusce est. Vivamus a tellus. Pellentesque habitant morbi tristique senectus [\dots] fames ac turpis egestas. Proin pharetra nonummy pede. \cite[S. 6]{Meier:Globalisierung}}
	\end{addmargin}
	
	\item	„Ein Plagiat ist nicht nur ein wörtliches Zitat ohne Anführungszeichen, sondern auch ein sinngemäßes indirektes Zitat, das den Anschein erweckt, es sei aus eigenen Erkenntnissen entstanden.“\footnote{Ebd.}
	
	\item	Wörtliche Zitate sollten nur im Ausnahmefall unmittelbar einer Überschrift folgen.
	
	\item	Bei Zitaten dürfen keine inhaltlichen Veränderungen vorgenommen werden.
	
	\item	Alle Aussagen ohne Quellenangabe stellen entweder eigene Standpunkte, eigene Erkenntnisse oder generell anerkannte Tatsachen dar.
	
	\item	Übersetzungen müssen in der Fußnote kenntlich gemacht werden.
	
	\item	Zahlen haben nur dann einen wissenschaftlichen Wert, wenn sie nachprüfbar sind. Daher ist immer eine Quellenangabe bei Zahlenangaben erforderlich.
	
\end{itemize}

\subsection{Wörtliche und sinngemäße Zitate}
\begin{itemize}
	\item	Wörtlich übernommene Texte sind immer in Anführungszeichen zu setzen, außer es handelt sich um ein Blockzitat [siehe oben]. Verwenden Sie stets die deutschen doppelten Anführungszeichen („ “) und nicht die englischen (“ ”) Anführungszeichen zur Kennzeichnung eines Zitates. Enthält ein deutsches Zitat bereits Anführungszeichen, so ändern Sie diese in einfache deutsche Anführungszeichen (‚  ‘) um:
	\begin{addmargin}[10pt]{0pt}
		\textcolor{red}{Müller schreibt: „In früheren Zeiten hatte die Obrigkeit die Finanzgewalt in ihrer ‚festen‘ Hand“ [Müller, 2010, S. 5].}
	\end{addmargin}
	\item	Eigene Ergänzungen oder Veränderungen sind in eckige Klammern zu setzen. Dabei darf der Inhalt des Zitats nicht verändert werden:
	\begin{addmargin}[10pt]{0pt}
		\textcolor{red}{Nach Auffassung von Müller „[hatte] in früheren Zeiten die Obrigkeit die Finanzgewalt [\dots]“ [Müller, 2010, S. 5].}
	\end{addmargin}
	\item	Auslassungen eines Wortes oder mehrerer Wörter sind durch 3 Punkte in eckigen Klammern anzudeuten. Die Auslassungspunkte sind über den Befehl \verb|\dots| zu erzeugen (und nicht durch drei Satzpunkte!):
	\begin{addmargin}[10pt]{0pt}
		\textcolor{red}{„In früheren Zeiten hatte die Obrigkeit […]“ [Müller, 2010, S. 5].}
	\end{addmargin}
	\item	Bei sinngemäßer Wiedergabe erfolgt lediglich die Angabe der Quelle ohne Anführungszeichen:
	\begin{addmargin}[10pt]{0pt}
		\textcolor{red}{Nach Auffassung von Müller besaß die Obrigkeit Kontrolle über die Finanzen [vgl. Müller, 2010, S. 5].}
	\end{addmargin}
	\item	Ein wiederholtes Zitieren derselben Quelle kann mit \textsl{ebenda} durchgeführt werden. Es muss dabei ein direkter Quellenbezug gegeben sein:
	\begin{addmargin}[10pt]{0pt}
		\textcolor{red}{Nach Auffassung von Müller besaß die Obrigkeit Kontrolle über die Finanzen [vgl. Müller, 2010, S. 5]. Die Unterschicht hatte keine Macht  [vgl. ebd., S. 7].}
	\end{addmargin}
\end{itemize}

\subsection{Unterschiedliche Zitierweisen}
\label{sec: ZitierStil}
Sprechen Sie mit Ihrem*Ihrer Betreuer*in die zu verwendende Zitierweise ab.

\subsubsection{Quellangaben im Text (AGR, AT, MB, PDK)}
Meier behauptet, dieser Aspekt sei wichtig~\citeauthoryear[vgl.][S.~5]{Meier:Globalisierung}.\newline
Meier meint, „dieser Punkt […] ist relevant“~\citeauthoryear[S.~5]{Meier:Globalisierung}.\newline
Dies entspricht auch der neueren Literatur~\citeauthoryear[vgl.][S.~10]{Mueller:Meier}.\newline
Vor Kurzem wurde ein Überblick über den aktuellen Forschungsstand zum Thema veröffentlicht~\citeauthoryear[vgl.][S.~20-35]{Mueller:Meier:Huber}.\newline
In früheren wissenschaftlichen Veröffentlichungen war dieses Thema als Forschungslücke angesehen worden~\citeauthoryear[vgl.][S.~85]{Mueller:Meier:Huber:Tausch}.\newline
\underline{BibStyle}: \textsf{FHWelsAlphabeticBrackets}


\subsubsection{Quellangaben im Text (BI, BUT, IPM, LCW, LTE)}
Meier behauptet, dieser Aspekt sei wichtig (Meier, 2011, S. 5).\newline
Meier meint, „dieser Punkt […] ist relevant“ (Meier, 2011, S. 5).\newline
Dies entspricht auch der aktuelleren Literatur (Müller und Meier, 2019, S. 10).\newline
Vor Kurzem wurde ein Überblick über den aktuellen Forschungsstand zum Thema veröffentlicht 
(Müller, Meier und Huber, 2021, S. 20-35).\newline
In früheren wissenschaftlichen Veröffentlichungen war dieses Thema als Forschungslücke angesehen worden (Müller et al., 2016, S. 85).\newline
\underline{BibStyle}: Diese Zitierweise ist nicht in der Latex-Vorlage abgebildet. Zum Zitieren mit dem
\verb|\cite|-Befehl sind die Style-Files entsprechend anzupassen!


\subsubsection{Quellangaben in der Fußnote (AMM, MEWI)}
Meier behauptet, dieser Aspekt sei wichtig.\footciteauthoryear[Vgl.][S.~5]{Meier:Globalisierung}\newline
Meier meint, „dieser [\dots] Punkt ist relevant“.\footciteauthoryear[S.~5]{Meier:Globalisierung}\newline
Dies entspricht auch der aktuelleren Literatur.\footciteauthoryear[Vgl.][S.~10]{Mueller:Meier}\newline
Vor Kurzem wurde ein Überblick über den aktuellen Forschungsstand zum Thema veröffentlicht.\footciteauthoryear[Vgl.][S.~20-35]{Mueller:Meier:Huber}\newline 
In früheren wissenschaftlichen Veröffentlichungen war dieses Thema als Forschungslücke angesehen worden.
\footciteauthoryear[Vgl.][S.~85]{Mueller:Meier:Huber:Tausch}\newline  
\underline{BibStyle}: \textsf{FHWelsAlphabeticBrackets}

\subsubsection{Quellangaben im Text mit Zahlen (AB, AET, AGR, AMM, AT, BUT, LTE, MB, MEWI, RSE, SES, VTP, WFT)}
Dieser Aspekt ist wichtig~\cite[vgl.][S. 5]{Meier:Globalisierung}.\newline
Nach \cite[S. 5]{Meier:Globalisierung} ist dieser Aspekt wichtig.\newline
Meier behauptet, dieser Aspekt sei wichtig~\cite[vgl.][S.~10]{Mueller:Meier}.\newline
Meier meint, „dieser […] Punkt ist relevant“~\cite[S.~5]{Mueller:Meier}.\newline
Dies entspricht auch der aktuelleren Literatur~\cite[vgl.][S.~10]{Mueller:Meier:Huber}.\newline
Vor Kurzem wurde ein Überblick über den aktuellen Forschungsstand zum Thema veröffentlicht~\cite[vgl.][S.~20-35]{Mueller:Meier:Huber:Tausch}.\newline
In früheren wissenschaftlichen Veröffentlichungen war dieses Thema als Forschungslücke angesehen worden~\cite[vgl.][S.~85]{Mueller:Meier:Huber:Tausch}.\newline
\underline{BibStyle}: \textsf{FHWelsNumericBrackets}

\subsubsection{Zitieren mit \LaTeX}
Die in dieser Vorlage verwendeten Zitierstile können von den Standardstilen \textsf{numeric} und \textsf{authoryear} abgeleitet werden. Dadurch ist es auch möglich, die unterschiedlichen Darstellungsarten zum Zitieren zu verwenden. Genauere Informationen können der Beschreibung des \textit{BibLatex}-Pakets (\url{https://ctan.org/pkg/biblatex?lang=de}) entnommen werden.\newline
\\Die wichtigsten Befehle:
\begin{itemize}
	\item \verb|\cite{}|:\\ Standardbefehl (\zB: [Meier, 2011])
	\item \verb|\cite[pre][post]{}|:\\ erweiterter Standardbefehl (\zB: [<pre> Meier, 2011, <post>])
	\item \verb|\citeauthor|:\\ nur Autor der Quelle (\zB: Meier)
	\item \verb|\footcite{}|:\\ zitieren als Fußnote
	\item \verb|\footcite[pre][post]{}|:\\ zitieren als Fußnote mit Erweiterungen (siehe \verb|\cite[pre][post]{}|)
\end{itemize}

