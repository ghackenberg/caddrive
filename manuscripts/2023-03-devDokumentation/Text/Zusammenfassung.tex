%LTeX: language=de-DE
\chapter{Zusammenfassung}
Bei dieser Software handelt es sich um eine ''Fullstack'' Webapplikation die komplett mit ''TypeScript'' erstellt wurde. In der nachfolgenden Liste sind zusammenfassend nochmal die wichtigsten Technologien aufgelistet, welche von dieser Applikation verwendet werden:

\begin{itemize}
    \item Übergreifend: TypeScript, Auth0, MQTT
    \item Gateway: Express
	\item Frontend: React.js, Axios, Three.js, Recharts
	\item Backend: Nest.js, Swagger, Postman
	\item Broker: Aedes,
	\item Datenbank: Docker, PostgreSQL, pgAdmin, TypeORM
\end{itemize}

Die Anwendung bietet noch eine Vielzahl von möglichen Funktionen, um die sie erweitert werden kann, wie beispielsweise:

\begin{itemize}
    \item Einbinden von Requirements Engineering
    \item Einbinden der Überführung von Requirements in eine CAD-feie Produktstruktur
    \item Das Verwalten von Produkten in verschiedensten Ausprägungen
    \item Einbauen von Simulationstools wie zum Beispiel ''FEM'' oder ''CFD''
    \item Prüfung der Qualität der CAD-Modelle anhand von Checklisten oder Meetings
    \item Umstellung von ''Three.js'' auf eine andere Engine für mehr Performance beim Rendern der Modelle
    \item Integration von verschiedensten Formaten oder Tools wie: PDF, Audio, Video, Simulationen oder CAD-Konstruktionstools für einen breiteren Einsatzbereich
\end{itemize}

Ein großer Vorteil dieses Projekts ist, dass durch die Anzahl an Technologien und durch die Abwechslung der Tätigkeiten viele neue Erkenntnisse gewonnen werden. Weiters werden nur modernste Technologien eingesetzt, die auch in der Wirtschaft breite Anwendung finden.