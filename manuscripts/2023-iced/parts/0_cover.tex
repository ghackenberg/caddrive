\title{GitHub for product development -- How could that look like?}

\author{
    \IEEEauthorblockN{Georg Hackenberg}
    \IEEEauthorblockA{
        School of Engineering, University of \\
        Applied Sciences Upper Austria \\
        4600 Wels, Upper Austria, Austria \\
        \href{mailto:georg.hackenberg@fh-wels.at}{georg.hackenberg@fh-wels.at}
    }
    \and
    \IEEEauthorblockN{Dominik Frühwirth}
    \IEEEauthorblockA{
        School of Engineering, University of \\
        Applied Sciences Upper Austria \\
        4600 Wels, Upper Austria, Austria \\
        \href{mailto:dominik.fruewirth@fh-wels.at}{dominik.fruewirth@fh-wels.at}
    }
    \and
    \IEEEauthorblockN{Christian Zehetner}
    \IEEEauthorblockA{
        School of Engineering, University of \\
        Applied Sciences Upper Austria \\
        4600 Wels, Upper Austria, Austria \\
        \href{mailto:christian.zehetner@fh-wels.at}{christian.zehetner@fh-wels.at}
    }
}

\maketitle

\begin{abstract}
    Product development is facing new challenges due to increasingly complex and individualized products in small batch sizes and short time to markets at high quality standards.
    Integrated product data management along with systematic requirements engineering and early stakeholder involvement are known to be key enablers for the success of future product development.
    In software development, established platforms such as GitHub exist, which have been shown to improve stakeholder communication, requirements elicitation, and software design decisions.
    In product development, similar platforms exist with impressive functionality, but which have some drawbacks such as closed source licenses, vendor-specific data formats, and expert-level user interfaces.
    To overcome the current situation, we study how the ideas of GitHub can be translated to an open source solution for product development and which concepts can be reused or must be changed.
    Core deliverables of our work are (1) an integrated data model of requirements (or design tasks), project schedules, and revisions of computer-aided design (CAD) models as well as (2) an interface model.
\end{abstract}

\begin{IEEEkeywords}
    Product development, computer-aided design, requirements engineering, project management
\end{IEEEkeywords}