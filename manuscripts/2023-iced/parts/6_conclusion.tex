\section{Conclusion}
\label{sec:conclusion}

The topic of product development and especially requirement engineering is very well covered by literature.
Most articles deal with theoretical concepts on how product development can be improved.
According to our literature research there are many excellent concepts but only few software tools that solve the problems practically.
Thus, we have developed a software tool that covers aspects such as version, issue and milestone management to provide an integrated platform for agile CAD.
Exports from 3D CAD models can be uploaded and managed with our version management.
With built-in tools such as issues, comments and milestones, agile product design can be performed together with the customer.
We have tested and critically evaluated our software.
Thereby a few questions came up which we discuss in the following.

\subsection{Model deltas}

Existing platforms for software development like GitHub provide the ability to automatically detect changes that have been made to source code.
This ability helps the software engineers to quickly understand what others have been doing and how that might affect their own work.
It remains an open question how such techniques can be translated effectively to our tooling, especially when you don't want to share every detail with the customer.

\subsection{Product variants}

Then, at the moment we only deal with versions of the same product, but we do not really cover the case of variant design.
If you want to cover the case of variant design properly, you probably need to introduce concepts like product platforms, modularization, and configuration.
Platforms typically represent design elements which are common to all variants, while modules and their configuration represent the individual selections that have been made by the customers.

\subsection{Simulation results}

Furthermore, it would be interesting to include simulation results into the platform for verification and validation purposes.
For example, product designers can use simulation results to prove certain dynamic product characteristics to the customer.
At the same time, customers can use the simulation results to validate their requirement formulations and rephrase them in case the original formulations are incorrect or misleading.

\subsection{Test automations}

Finally, it would be great to have the possibility to automate certain test procedures when new CAD model revisions are uploaded.
Consequently, the quality of the product design could be monitored more seamlessly and design flaws could be detected earlier.
If these test automations are connected to design tasks, then the resolution of design tasks could be bound to the condition that the test automations are passed successfully.